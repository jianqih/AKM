\documentclass{article}
\usepackage{graphicx}
\usepackage{amsmath}
\usepackage{amssymb}
\usepackage{setspace}
\usepackage{geometry}
\geometry{a4paper, margin=1in}
\usepackage{listings}   
\usepackage{xcolor}
\usepackage{mathpazo}
\lstset{
    language=[LaTeX]TeX,
    basicstyle=\ttfamily,
    breaklines=true,
    frame=single,
    numbers=left,
    numberstyle=\tiny\color{gray},
    tabsize=2,
    showspaces=false,
    showstringspaces=false
}

\begin{document} 

\title{AKM Model}
\author{2025-05-24}
\date{}
\maketitle

% The AKM model is a model of job matching and worker mobility. It is named after the authors Adam, Keuschnigg, and Maskin. 

\section{The Model}
From a standard human capital wage regression like 
$$ \ln w_{it} = \beta x_{it} + \epsilon _i t $$
explains about 30\% of wage variation.

In a series of papers, AKM introduce
\begin{equation}
    y_{it} = \alpha _i + \psi _{j(i,t)} + x_{it} \beta + \epsilon_{it}
\end{equation}
where $j(i,t)$ is a random variable that takes a value in $\{1,2,\dots,J_t\}$ with probability $p_{jt}$ and $p_{jt}$ is a function of $\alpha_i$ and $\psi_j$.

% In our code, we preset the number of firms and worker types. 
In AKM's setting, $y_{it}$ is the outcome of person $i$ and observable, $x_{it}$ the observable characteristics of person $i$ at time $t$, 
$\alpha_i$ the unobservable characteristics of person $i$, $\psi_j$ the unobservable characteristics of firm $j$, and $\epsilon_{it}$ the idiosyncratic shock.

In our model, we preset the unobservable in order to get the characteristics of the workers and firms. 
The output: $G$ is the transition matrix of workers. $H$ is the joint distribution of workers and firms. 
They calculate the transition probability matrix $G$ by 
$$ \psi - c_{net} \psi_j - c_{sort} \alpha_i $$
where $c_{net}$ is the network effect and $c_{sort}$ is the sorting effect. 



\end{document}

